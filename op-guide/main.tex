\documentclass[7pt,a5paper]{book}

% Language setting
% Replace `english' with e.g. `spanish' to change the document language
\usepackage[english]{babel}

\usepackage[margin=1in]{geometry}

\renewcommand{\familydefault}{\sfdefault}

\usepackage{graphicx}
\usepackage{hyperref}

\usepackage[document]{ragged2e}

\usepackage[acronym]{glossaries}

% \makeglossaries

\newacronym{pcr}{PCR}{Production Control Room}
\newacronym{scr}{SCR}{Sound Control Room}
\newacronym{vcr}{VCR}{Vision Control Room}

\title{
\includegraphics[width=0.5\textwidth]{../uos-logo.png}\\
Television Studio\\Operations Guide
}
\author{Susan Pratt \thanks{\href{mailto:s.pratt@surrey.ac.uk}{s.pratt@surrey.ac.uk}} \and Sam Sarjudeen \thanks{\href{mailto:s.sarjudeen@surrey.ac.uk}{s.sarjudeen@surrey.ac.uk}}}

\begin{document}
\maketitle

\chapter{Operational Responsibilities}

\newpage
\section{Studio Director}
% \input{op-resp/director}
\begin{itemize}
    \item Decides format of the programme with the producer
    \item Creates the camera shots and positions for each part of the programme
    \item Works with the Lighting Director to decide on the appropriate lighting for show
    \item Works with the Sound Supervisor to discuss sound, effects, music and comms for the show
    \item Works with set designer or if an existing show uses existing set to best use for particular programme
    \item Creates running order with the PA and Producer and creates the call sheet
    \item Co-ordinates with graphics and VT play-in
    \item Rehearses the studio before the show and decides on camera position and framing with camera team
    \item Directs the vision mixer during rehearsal and leads them during the show
    \item Follows the script and running order of the programme
    \item Decides on the next shot, VT, graphic or other source to be used next
    \item Stands by VT before running VT
    \item Warns the camera or graphics before Vision mixer cuts to it
    \item Stands by and cues presenter (sometimes PA)
    \item Cues digital video effects or animations on graphics
    \item Directs the cap-gens when called for from count from PA or from the script
    \item Informs the studio team of any changes in the running order which may be as a result of editorial changes from the producer
    \item Is in charge of the technical side of the studio
    \item Is responsible for the smooth running of the programme
    \item Takes technical decisions on the programme on advice from the technical manager
\end{itemize}

\newpage
\section{Producer}
% \input{op-resp/producer}
\begin{itemize}
    \item In charge of content of the programme
    \item Works with the director to create the look of the programme
    \item Has editorial control of the order of the programme content
    \item Director can overrule producer on technical grounds (normally)
    \item Talks to presenter on talk back
    \item Discusses timings of the programme with the PA and makes suitable adjustments
    \item Producer allocates assistant producer responsibility to stories or parts of stories
    \item Reporter with assistant producer on packages for Producer
    \item Researchers provide research on stories for producer which is used by presenters, reporters or assistant producers
\end{itemize}

\newpage
\section{PA}
The Production Assistant (PA) or Programme Coordinator (PC)...
\begin{itemize}
    \item Works to the programme duration for live or recording
    \item Tests communications to the presenter
    \item Checks whether presenter is working on open or switched talkback
    \item Works to running order which has all the durations of VTs, reads and titles
    \item Keep track of programme timing and keep the producer informed whether there is an under-run or over-run
    \item Advises producer to make adjustments to get the timing back on track
    \item Times each of the items and informs the studio over talkback of the timing left on VT or interview or other live event
    \item Gives studio countdown to transmission or recording: 2 mins, 1 min, 30s, 20s, hard count 10-0 and then notes if the broadcast is on-air if it is live
    \item Gives director timings on VTs and graphics
    \item Keeps paperwork and music copyright for Programme as Braodcast (PAB)
\end{itemize}

\newpage
\section{Presenter (Talent)}
\begin{itemize}
    \item Listens to the director over talkback (if available)
    \item Listens to the producer over talkback (if available)
    \item Talks to director through their microphone if it is open or speaks to the floor manager who will relay to director
    \item Has scripts ready to read and have researched subject and questions for interviews
    \item Will have make-up completed by the agreed call time
    \item Be in studio by agreed call time
    \item Put on talkback unit and earpiece if one is available
    \item Hears directions over talkback and/or visually from floor manager
    \item Takes visual cue from floor manager to start reading script
    \item Takes visual cue for the camera from floor manager
    \item Follow PA talkback or floor manager's hand signals to shorter or longer duration item
    \item Quiet by the time that they have been ask to, PA will count to stopping talking
    \item Pleasant to all technical staff
\end{itemize}

\newpage
\section{Floor Manager}
\begin{itemize}
    \item Listens to the director over talkback
    \item Is the eyes and ears of the director on the studio floor is responsible for health and safety on the studio floor
    \item Is in charge of the studio floor and co-ordinates with all personnel in that area
    \item Speaks loudly and clearly
    \item Normally 30 sec warning for rehearsal or 2 mins warning for live recording
    \item Uses hand signals to give timing to on-air, off-air and timings of items and interviews
    \item Verbal warning to transmission/recording 2 mins, 1 min, 30s, 20s, 10, 9, 8, 7, 6, 5 and 4s, only hand signals for 3, 2 and 1s. (knows the hand signals for all the timings)
    \item Knows hand signals to cue presenter and extend or shorten speech
\end{itemize}

\newpage
\section{Camera Operator}









\chapter{Operational Guides}
\section{General}
\subsection{Comms}
\section{Studio Floor}
\section{Production Control Room}
The \gls{pcr}
\subsection{Producer}
\subsection{PA}
\subsection{Director}
\subsection{Vision Mixer}
\subsection{Engineer}
\subsection{Autocue Operator}
\subsection{GFX Operator}
\subsection{EVS Operator}
\section{Sound Control Room}
\subsection{Sound Mixer}
\subsection{Gram Operator/Comms Engineer}
This multipurpose position...
\section{Vision Control Room}

\end{document}