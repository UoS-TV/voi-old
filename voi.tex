\documentclass{report}


% \usepackage[utf8]{inputenc}
% Language setting
% Replace `english' with e.g. `spanish' to change the document language
\usepackage[english]{babel}

% Set page size and margins
% Replace `letterpaper' with `a4paper' for UK/EU standard size
\usepackage[a4paper,margin=1in]{geometry}

\renewcommand{\familydefault}{\sfdefault}

\addto\captionsenglish{\renewcommand\chaptername{VOI}}

\usepackage{booktabs}
\usepackage{longtable}

\usepackage{graphicx}


\usepackage{amssymb}% for the checkmark

\usepackage[acronym]{glossaries}

\usepackage{hyperref}

% \makeglossaries

\newacronym{pcr}{PCR}{Production Control Room}
\newacronym{scr}{SCR}{Sound Control Room}
\newacronym{vcr}{VCR}{Vision Control Room}
\newacronym{tar}{TAR}{Technical Apparatus Room}

\title{
\includegraphics[width=0.5\textwidth]{uos-logo.png}\\\vspace{1cm}
Film Production and Broadcast Engineering\\Vidmeister Operating Instructions
}
\author{Alan Haigh \thanks{\href{mailto:a.haigh@surrey.ac.uk}{a.haigh@surrey.ac.uk}} \and Susan Pratt \thanks{\href{mailto:s.pratt@surrey.ac.uk}{s.pratt@surrey.ac.uk}} \and Sam Sarjudeen \thanks{\href{mailto:s.sarjudeen@surrey.ac.uk}{s.sarjudeen@surrey.ac.uk}}}



\begin{document}

\maketitle
\tableofcontents
% \chapter*{Introduction}

\chapter{Bookable Items and Rooms}

The department uses an online booking system for all bookable equipment and rooms. The tables in \ref{stock} and \ref{rooms} show all bookable stock items and rooms depending on year group. 3rd years (i.e. those on placement) aren't able to book on the system and, should they need to book anything while on placement, are encouraged to email DMM Stores or a technician who will arrange the booking with them.

Information correct as of \today.

\section{Stock}\label{stock}

\begin{longtable}{p{0.6\textwidth}ccc}
\caption{Camcorders} \\
\toprule
Asset Name & Year 1 & Year 2 & Year 4 \\
\midrule
\endfirsthead
\caption[]{Camcorders} \\
\toprule
Asset Name & Year 1 & Year 2 & Year 4 \\
\midrule
\endhead
\midrule
\multicolumn{4}{r}{Continued on next page} \\
\midrule
\endfoot
\bottomrule
\endlastfoot
Panasonic UX Kit & \checkmark & \checkmark & \checkmark \\
Sony FS5 & \checkmark & \checkmark & \checkmark \\
UltraSync One & \checkmark & \checkmark & \checkmark \\
UltraSync One No.2 & \checkmark & \checkmark & \checkmark \\
Colorchecker PASSPORT VIDEO &  & \checkmark & \checkmark \\
Sony F5 Camera Kit &  & \checkmark & \checkmark \\
Sony F55 Camera &  & \checkmark & \checkmark \\
\end{longtable}
\begin{longtable}{p{0.6\textwidth}ccc}
\caption{Camcorders, DSLR Cameras} \\
\toprule
Asset Name & Year 1 & Year 2 & Year 4 \\
\midrule
\endfirsthead
\caption[]{Camcorders, DSLR Cameras} \\
\toprule
Asset Name & Year 1 & Year 2 & Year 4 \\
\midrule
\endhead
\midrule
\multicolumn{4}{r}{Continued on next page} \\
\midrule
\endfoot
\bottomrule
\endlastfoot
Sony FX3 Accessories Kit & \checkmark & \checkmark & \checkmark \\
Sony FX3 Cinema Camera & \checkmark & \checkmark & \checkmark \\
\end{longtable}
\begin{longtable}{p{0.6\textwidth}ccc}
\caption{Camera and Lens Accessories} \\
\toprule
Asset Name & Year 1 & Year 2 & Year 4 \\
\midrule
\endfirsthead
\caption[]{Camera and Lens Accessories} \\
\toprule
Asset Name & Year 1 & Year 2 & Year 4 \\
\midrule
\endhead
\midrule
\multicolumn{4}{r}{Continued on next page} \\
\midrule
\endfoot
\bottomrule
\endlastfoot
AC Bag &  & \checkmark & \checkmark \\
Arri Follow Focus Kit &  & \checkmark & \checkmark \\
Sony FS5 Raincover (Rain Slicker) &  & \checkmark & \checkmark \\
\end{longtable}
\begin{longtable}{p{0.6\textwidth}ccc}
\caption{DSLR Cameras} \\
\toprule
Asset Name & Year 1 & Year 2 & Year 4 \\
\midrule
\endfirsthead
\caption[]{DSLR Cameras} \\
\toprule
Asset Name & Year 1 & Year 2 & Year 4 \\
\midrule
\endhead
\midrule
\multicolumn{4}{r}{Continued on next page} \\
\midrule
\endfoot
\bottomrule
\endlastfoot
Canon 6D  & \checkmark & \checkmark & \checkmark \\
\end{longtable}
\begin{longtable}{p{0.6\textwidth}ccc}
\caption{DSLR Cameras, Tripods {\&} Mounts} \\
\toprule
Asset Name & Year 1 & Year 2 & Year 4 \\
\midrule
\endfirsthead
\caption[]{DSLR Cameras, Tripods {\&} Mounts} \\
\toprule
Asset Name & Year 1 & Year 2 & Year 4 \\
\midrule
\endhead
\midrule
\multicolumn{4}{r}{Continued on next page} \\
\midrule
\endfoot
\bottomrule
\endlastfoot
Dual Grip Camera Should Rig & \checkmark & \checkmark & \checkmark \\
Nucleus-Nano Wireless Lens Control System & \checkmark & \checkmark & \checkmark \\
\end{longtable}
\begin{longtable}{p{0.6\textwidth}ccc}
\caption{Filters} \\
\toprule
Asset Name & Year 1 & Year 2 & Year 4 \\
\midrule
\endfirsthead
\caption[]{Filters} \\
\toprule
Asset Name & Year 1 & Year 2 & Year 4 \\
\midrule
\endhead
\midrule
\multicolumn{4}{r}{Continued on next page} \\
\midrule
\endfoot
\bottomrule
\endlastfoot
Filter Schneider 4x5.65 - 85 &  & \checkmark & \checkmark \\
Filter Schneider 4x5.65 - Amber Sev 1 &  & \checkmark & \checkmark \\
Filter Schneider 4x5.65 - Clear &  & \checkmark & \checkmark \\
Filter Schneider 4x5.65 - Coral 1/8 &  & \checkmark & \checkmark \\
Filter Schneider 4x5.65 - HD Classic Soft 1/4 &  & \checkmark & \checkmark \\
Filter Schneider 4x5.65 - Lin Trued Pol &  & \checkmark & \checkmark \\
Filter Schneider 4x5.65 - ND 0.6 SEH &  & \checkmark & \checkmark \\
Filter Schneider 4x5.65 - True-Streak Blue  2MM &  & \checkmark & \checkmark \\
Filter Tiffen Black Pro-mist 1/4 &  & \checkmark & \checkmark \\
\end{longtable}
\begin{longtable}{p{0.6\textwidth}ccc}
\caption{Headphones and Headphone amps} \\
\toprule
Asset Name & Year 1 & Year 2 & Year 4 \\
\midrule
\endfirsthead
\caption[]{Headphones and Headphone amps} \\
\toprule
Asset Name & Year 1 & Year 2 & Year 4 \\
\midrule
\endhead
\midrule
\multicolumn{4}{r}{Continued on next page} \\
\midrule
\endfoot
\bottomrule
\endlastfoot
Beyerdynamic DT770 Headphones No. 1 &  & \checkmark & \checkmark \\
Beyerdynamic DT770 Headphones No. 2 &  & \checkmark & \checkmark \\
\end{longtable}
\begin{longtable}{p{0.6\textwidth}ccc}
\caption{Lenses} \\
\toprule
Asset Name & Year 1 & Year 2 & Year 4 \\
\midrule
\endfirsthead
\caption[]{Lenses} \\
\toprule
Asset Name & Year 1 & Year 2 & Year 4 \\
\midrule
\endhead
\midrule
\multicolumn{4}{r}{Continued on next page} \\
\midrule
\endfoot
\bottomrule
\endlastfoot
Samyang E-Mount Lens kit & \checkmark & \checkmark & \checkmark \\
Sigma 18-35 mm T2 Zoom &  & \checkmark & \checkmark \\
Sigma 50-100mm T2 Zoom &  & \checkmark & \checkmark \\
Sony CineAlta Six Lens Kit - PL Mount &  & \checkmark & \checkmark \\
\end{longtable}
\begin{longtable}{p{0.6\textwidth}ccc}
\caption{Lenses, Lighting Accessories} \\
\toprule
Asset Name & Year 1 & Year 2 & Year 4 \\
\midrule
\endfirsthead
\caption[]{Lenses, Lighting Accessories} \\
\toprule
Asset Name & Year 1 & Year 2 & Year 4 \\
\midrule
\endhead
\midrule
\multicolumn{4}{r}{Continued on next page} \\
\midrule
\endfoot
\bottomrule
\endlastfoot
Aputure F10 Fresnel Lens & \checkmark & \checkmark & \checkmark \\
\end{longtable}
\begin{longtable}{p{0.6\textwidth}ccc}
\caption{Lighting} \\
\toprule
Asset Name & Year 1 & Year 2 & Year 4 \\
\midrule
\endfirsthead
\caption[]{Lighting} \\
\toprule
Asset Name & Year 1 & Year 2 & Year 4 \\
\midrule
\endhead
\midrule
\multicolumn{4}{r}{Continued on next page} \\
\midrule
\endfoot
\bottomrule
\endlastfoot
Aputure Light Storm 600D Pro LED Light & \checkmark & \checkmark & \checkmark \\
Aputure MC 4-Light Travel Kit & \checkmark & \checkmark & \checkmark \\
Arri 2000W Head Kit & \checkmark & \checkmark & \checkmark \\
Arri 600W 3 head kit & \checkmark & \checkmark & \checkmark \\
Arri 650W 3 head kit & \checkmark & \checkmark & \checkmark \\
Arri 800W 3 Head Kit & \checkmark & \checkmark & \checkmark \\
Arri Lighting Kit & \checkmark & \checkmark & \checkmark \\
Dedo LED Light Kit & \checkmark & \checkmark & \checkmark \\
Dedo Light Kit & \checkmark & \checkmark & \checkmark \\
Dedo Light Projector kit & \checkmark & \checkmark & \checkmark \\
Hot-Shoe LED light & \checkmark & \checkmark & \checkmark \\
Lishuai LED Panel & \checkmark & \checkmark & \checkmark \\
Rotolight Neo 3 Light Kit & \checkmark & \checkmark & \checkmark \\
\end{longtable}
\begin{longtable}{p{0.6\textwidth}ccc}
\caption{Lighting Accessories} \\
\toprule
Asset Name & Year 1 & Year 2 & Year 4 \\
\midrule
\endfirsthead
\caption[]{Lighting Accessories} \\
\toprule
Asset Name & Year 1 & Year 2 & Year 4 \\
\midrule
\endhead
\midrule
\multicolumn{4}{r}{Continued on next page} \\
\midrule
\endfoot
\bottomrule
\endlastfoot
Aputure F10 Barn Doors & \checkmark & \checkmark & \checkmark \\
Arri Reflector - Medium  & \checkmark & \checkmark & \checkmark \\
Lastolite Reflector - Large & \checkmark & \checkmark & \checkmark \\
Lastolite Reflector - Medium & \checkmark & \checkmark & \checkmark \\
Lastolite Reflector - Small & \checkmark & \checkmark & \checkmark \\
Lastolite SkyRapid Fabric Diffuser Square & \checkmark & \checkmark & \checkmark \\
Neewer 5-in-1 Collapsible Light Reflector 110cm & \checkmark & \checkmark & \checkmark \\
Octagonal SoftBox & \checkmark & \checkmark & \checkmark \\
ChromaFlex And LiteRing &  & \checkmark & \checkmark \\
Lastolite EzyBalance &  & \checkmark & \checkmark \\
Lastolite EzyBalance &  & \checkmark & \checkmark \\
\end{longtable}
\begin{longtable}{p{0.6\textwidth}ccc}
\caption{Location Video Monitors} \\
\toprule
Asset Name & Year 1 & Year 2 & Year 4 \\
\midrule
\endfirsthead
\caption[]{Location Video Monitors} \\
\toprule
Asset Name & Year 1 & Year 2 & Year 4 \\
\midrule
\endhead
\midrule
\multicolumn{4}{r}{Continued on next page} \\
\midrule
\endfoot
\bottomrule
\endlastfoot
JVC 9" Location Monitor &  & \checkmark & \checkmark \\
Location Floor Monitor &  & \checkmark & \checkmark \\
\end{longtable}
\begin{longtable}{p{0.6\textwidth}ccc}
\caption{Mains distribution} \\
\toprule
Asset Name & Year 1 & Year 2 & Year 4 \\
\midrule
\endfirsthead
\caption[]{Mains distribution} \\
\toprule
Asset Name & Year 1 & Year 2 & Year 4 \\
\midrule
\endhead
\midrule
\multicolumn{4}{r}{Continued on next page} \\
\midrule
\endfoot
\bottomrule
\endlastfoot
FVPT Mains Distribution no. 1 & \checkmark & \checkmark & \checkmark \\
FVPT Mains Distribution no. 2 & \checkmark & \checkmark & \checkmark \\
FVPT Mains Distribution no. 3 & \checkmark & \checkmark & \checkmark \\
\end{longtable}
\begin{longtable}{p{0.6\textwidth}ccc}
\caption{Mains extensions} \\
\toprule
Asset Name & Year 1 & Year 2 & Year 4 \\
\midrule
\endfirsthead
\caption[]{Mains extensions} \\
\toprule
Asset Name & Year 1 & Year 2 & Year 4 \\
\midrule
\endhead
\midrule
\multicolumn{4}{r}{Continued on next page} \\
\midrule
\endfoot
\bottomrule
\endlastfoot
FVPT Extension no. 1 & \checkmark & \checkmark & \checkmark \\
FVPT Extension no. 2 & \checkmark & \checkmark & \checkmark \\
FVPT Extension no. 3 & \checkmark & \checkmark & \checkmark \\
Single Mains Extension 1 & \checkmark & \checkmark & \checkmark \\
Single Mains Extension 2 & \checkmark & \checkmark & \checkmark \\
Single Mains Extension 3 & \checkmark & \checkmark & \checkmark \\
Single Mains Extension 4 & \checkmark & \checkmark & \checkmark \\
Single Mains Extension 5 & \checkmark & \checkmark & \checkmark \\
Single Mains Extension 6 & \checkmark & \checkmark & \checkmark \\
\end{longtable}
\begin{longtable}{p{0.6\textwidth}ccc}
\caption{Microphones} \\
\toprule
Asset Name & Year 1 & Year 2 & Year 4 \\
\midrule
\endfirsthead
\caption[]{Microphones} \\
\toprule
Asset Name & Year 1 & Year 2 & Year 4 \\
\midrule
\endhead
\midrule
\multicolumn{4}{r}{Continued on next page} \\
\midrule
\endfoot
\bottomrule
\endlastfoot
Rode Newsshooter Kit &  & \checkmark & \checkmark \\
Sennheiser MKH418S Shotgun Mic &  & \checkmark & \checkmark \\
FPBE AKG C414-XLS Kit 1 & \checkmark & \checkmark & \checkmark \\
FPBE AKG C414-XLS Kit 2 & \checkmark & \checkmark & \checkmark \\
Hebden Sound HS3000 x2 kit 3 & \checkmark &  &  \\
Hebden Sound HS3000 x2 kit 4 & \checkmark &  &  \\
Shure SM57 No. 1 & \checkmark &  &  \\
Shure SM57 No. 2 & \checkmark &  &  \\
Electrovoice RE20 kit 2 & \checkmark &  &  \\
Electrovoice RE20 kit 3 & \checkmark &  &  \\
Hebden Sound HS3000 x2 kit 1 & \checkmark &  &  \\
AKG SE300B & \checkmark & \checkmark & \checkmark \\
RODE NTG3B Shotgun Microphone & \checkmark & \checkmark & \checkmark \\
Rode Broadcaster & \checkmark & \checkmark & \checkmark \\
Rode Reporter Mic & \checkmark & \checkmark & \checkmark \\
Sennheiser ENG Wireless Mic Set 3 & \checkmark & \checkmark & \checkmark \\
Sennheiser ENG Wireless Mic Set 4 & \checkmark & \checkmark & \checkmark \\
Sony ECM-77B Lavalier (Tie-Clip) Mic & \checkmark & \checkmark & \checkmark \\
Schoeps CMIT 5U &  & \checkmark & \checkmark \\
Sennheiser ENG Wireless Mic Set 1 & \checkmark & \checkmark & \checkmark \\
Sennheiser ENG Wireless Mic Set 2 & \checkmark & \checkmark & \checkmark \\
Sennheiser MKH816T & \checkmark & \checkmark & \checkmark \\
\end{longtable}
\begin{longtable}{p{0.6\textwidth}ccc}
\caption{Performance equipment} \\
\toprule
Asset Name & Year 1 & Year 2 & Year 4 \\
\midrule
\endfirsthead
\caption[]{Performance equipment} \\
\toprule
Asset Name & Year 1 & Year 2 & Year 4 \\
\midrule
\endhead
\midrule
\multicolumn{4}{r}{Continued on next page} \\
\midrule
\endfoot
\bottomrule
\endlastfoot
Performance Sm58 No.1 & \checkmark &  &  \\
\end{longtable}
\begin{longtable}{p{0.6\textwidth}ccc}
\caption{Portable Recorders} \\
\toprule
Asset Name & Year 1 & Year 2 & Year 4 \\
\midrule
\endfirsthead
\caption[]{Portable Recorders} \\
\toprule
Asset Name & Year 1 & Year 2 & Year 4 \\
\midrule
\endhead
\midrule
\multicolumn{4}{r}{Continued on next page} \\
\midrule
\endfoot
\bottomrule
\endlastfoot
MixPre compact field mixer & \checkmark & \checkmark & \checkmark \\
Sound Devices MixPre-6 & \checkmark & \checkmark & \checkmark \\
Zoom H1 Handy Recorder & \checkmark & \checkmark & \checkmark \\
Ambient Clockit Controller &  & \checkmark & \checkmark \\
Atomos Shogun &  & \checkmark & \checkmark \\
Sound Devices 788T &  & \checkmark & \checkmark \\
Zoom F4 &  & \checkmark & \checkmark \\
\end{longtable}
\begin{longtable}{p{0.6\textwidth}ccc}
\caption{Stock} \\
\toprule
Asset Name & Year 1 & Year 2 & Year 4 \\
\midrule
\endfirsthead
\caption[]{Stock} \\
\toprule
Asset Name & Year 1 & Year 2 & Year 4 \\
\midrule
\endhead
\midrule
\multicolumn{4}{r}{Continued on next page} \\
\midrule
\endfoot
\bottomrule
\endlastfoot
Arri Sandbag No 1 & \checkmark & \checkmark & \checkmark \\
Matthews Floppy 48x48 Flag & \checkmark & \checkmark & \checkmark \\
Westcott Fast Flags & \checkmark & \checkmark & \checkmark \\
Hollyland MARS Pro Wireless Transmission &  & \checkmark & \checkmark \\
iLok Kit 1 &  & \checkmark & \checkmark \\
\end{longtable}
\begin{longtable}{p{0.6\textwidth}ccc}
\caption{Stock, Portable Recorders} \\
\toprule
Asset Name & Year 1 & Year 2 & Year 4 \\
\midrule
\endfirsthead
\caption[]{Stock, Portable Recorders} \\
\toprule
Asset Name & Year 1 & Year 2 & Year 4 \\
\midrule
\endhead
\midrule
\multicolumn{4}{r}{Continued on next page} \\
\midrule
\endfoot
\bottomrule
\endlastfoot
Ambient Lockit Box  &  & \checkmark & \checkmark \\
\end{longtable}
\begin{longtable}{p{0.6\textwidth}ccc}
\caption{Tripods {\&} Mounts} \\
\toprule
Asset Name & Year 1 & Year 2 & Year 4 \\
\midrule
\endfirsthead
\caption[]{Tripods {\&} Mounts} \\
\toprule
Asset Name & Year 1 & Year 2 & Year 4 \\
\midrule
\endhead
\midrule
\multicolumn{4}{r}{Continued on next page} \\
\midrule
\endfoot
\bottomrule
\endlastfoot
Boom Pole - 3m & \checkmark & \checkmark & \checkmark \\
C-Stand no1 & \checkmark & \checkmark & \checkmark \\
C-Stand no2 & \checkmark & \checkmark & \checkmark \\
DJI Ronin-MX & \checkmark & \checkmark & \checkmark \\
Glidecam XR-Pro & \checkmark & \checkmark & \checkmark \\
Hague DRIVE Camslide Slider & \checkmark & \checkmark & \checkmark \\
Manfrotto Fig-Rig - No. 02 & \checkmark & \checkmark & \checkmark \\
Manfrotto Nitrotech 608 Tripod & \checkmark & \checkmark & \checkmark \\
Miller Tripod Dollie & \checkmark & \checkmark & \checkmark \\
Panasonic UX Tripod & \checkmark & \checkmark & \checkmark \\
RHINO Slider EVO Pro  & \checkmark & \checkmark & \checkmark \\
RODE Boom Pole Pro {\#}01 & \checkmark & \checkmark & \checkmark \\
RODE Boom Pole Pro {\#}02 & \checkmark & \checkmark & \checkmark \\
Rode Boom Pole - 3m & \checkmark & \checkmark & \checkmark \\
Vinten Vision 5LF Tripod & \checkmark & \checkmark & \checkmark \\
Vinten Vision Blue Tripod & \checkmark & \checkmark & \checkmark \\
Indy Dolly System &  & \checkmark & \checkmark \\
Manfrotto 525MVB Tripod &  & \checkmark & \checkmark \\
Manfrotto Autople &  & \checkmark & \checkmark \\
Manfrotto Magic arm No1 &  & \checkmark & \checkmark \\
Manfrotto Magic arm No2 &  & \checkmark & \checkmark \\
Manfrotto Monitor Stand  &  & \checkmark & \checkmark \\
Miller Tripod  &  & \checkmark & \checkmark \\
Vinten PT 117 Tripod &  & \checkmark & \checkmark \\
Libec Tripod &  & \checkmark &  \\
\end{longtable}
\begin{longtable}{p{0.6\textwidth}ccc}
\caption{Vans} \\
\toprule
Asset Name & Year 1 & Year 2 & Year 4 \\
\midrule
\endfirsthead
\caption[]{Vans} \\
\toprule
Asset Name & Year 1 & Year 2 & Year 4 \\
\midrule
\endhead
\midrule
\multicolumn{4}{r}{Continued on next page} \\
\midrule
\endfoot
\bottomrule
\endlastfoot
Van A (BP17 YKA) &  & \checkmark & \checkmark \\
Van B (BP17 YKB) &  & \checkmark & \checkmark \\
\end{longtable}


\newpage
\section{Rooms}\label{rooms}

\begin{longtable}{p{0.2\textwidth}p{0.4\textwidth}ccc}
\caption{Rooms} \\
\toprule
Category & Asset Name & Year 1 & Year 2 & Year 4 \\
\midrule
\endfirsthead
\caption[]{Rooms} \\
\toprule
Category & Asset Name & Year 1 & Year 2 & Year 4 \\
\midrule
\endhead
\midrule
\multicolumn{5}{r}{Continued on next page} \\
\midrule
\endfoot
\bottomrule
\endlastfoot
Media Labs & Media Lab: Red (Nodus) & \checkmark & \checkmark & \checkmark \\
Media Labs & CMT Studio (13 AC 01) &  & \checkmark & \checkmark \\
Other & Audio Lab (01 AC 02) & \checkmark & \checkmark & \checkmark \\
Other & Lecture Theatre C (LTC) & \checkmark & \checkmark & \checkmark \\
PATS Edit Rooms & Edit 4 & \checkmark & \checkmark & \checkmark \\
PATS Edit Rooms & Edit 5 & \checkmark & \checkmark & \checkmark \\
PATS Edit Rooms & Edit 3 (25E) & \checkmark &  &  \\
TV Studio Rooms & TV Studio - All Control Rooms &  & \checkmark & \checkmark \\
TV Studio Rooms & TV Studio - Sound Control Room (07bAC01) &  & \checkmark & \checkmark \\
TV Studio Rooms & TV Studio - Studio Floor (07AC01) &  & \checkmark & \checkmark \\
TV Studio Rooms & TV Studio - Vision Control Room (07aAC01) &  & \checkmark & \checkmark \\
\end{longtable}


\chapter{Room Guides}

\section{Television Studio}
\subsection*{Startup and Shutdown Procedure}

These steps are for a part startup and shutdown. The TV Studio Technical Manager should be contacted for full startup and shutdown as there is critical infrastructure for DMM in the TV Studio. The shutdown process is the same as the startup procedure but in reverse.

\begin{enumerate}
    \item \gls{tar}
          \begin{enumerate}
              \item In the middle of the centre rack Power on MDU 4. This will turn on the CCUs.
              \begin{figure}[h]
                \centering
                \includegraphics[width=.5\linewidth]{uos-logo.png}
                \caption{CCU MDU}
              \end{figure}
              \item Underneath the CCU's, power on the MDU for the Sony vision mixer.
              \begin{figure}[h]
                \centering
                \includegraphics[width=.5\linewidth]{uos-logo.png}
                \caption{Vison Mixer MDU}
              \end{figure}
          \end{enumerate}
    \item \gls{pcr} (Front Row)
    \hypertarget{pos1}{}%
          \begin{enumerate}
              \item In the rack closest to the \gls{tar}, turn on the vision mixer panel.
              \begin{figure}[h]
                \centering
                \includegraphics[width=.5\linewidth]{uos-logo.png}
                \caption{Vison Mixer Panel}
              \end{figure}
              \item At the bottom of that rack, turn on the speaker amp.
              \begin{figure}[h]
                \centering
                \includegraphics[width=.5\linewidth]{uos-logo.png}
                \caption{Speaker Amp}
              \end{figure}
              \item Turn on any monitors that aren't on.
          \end{enumerate}
    \item \gls{pcr} (Back Row)
          \begin{enumerate}
            \item At the engineering position (far left), reach through the empty rack and switch on the mains switch on the rear panel to the right of the rack.
            \begin{figure}[h]
                \centering
                \includegraphics[width=.5\linewidth]{uos-logo.png}
                \caption{Rear Desk Power Switch}
              \end{figure}
          \end{enumerate}

    \item \gls{scr}
    \begin{enumerate}
        \item Behind the computer monitor to the right of the mixing desk, power on the sequential MDU. This will turn on all equipment in the room on sequentially to prevent any current overload or loud pops coming through the speakers.
        \begin{figure}[h]
            \centering
            \includegraphics[width=.5\linewidth]{uos-logo.png}
            \caption{\gls{scr} Sequential MDU}
          \end{figure}
    \end{enumerate}
\item \gls{vcr}
\begin{enumerate}
    \item Switch on the lighting desk using the power button on the rear right of the desk.
    \item Switch on the lighting monitor.
    \item Switch on the vision engineering monitor.
    \item Switch on the TV using the remote.
\end{enumerate}
\end{enumerate}









\subsection{Studio Floor}
\subsection{Production Control Room}
The \gls{pcr}
\subsubsection{Producer}
\subsubsection{PA}
\subsubsection{Director}
\subsubsection{Vision Mixer}
\subsubsection{Engineer}
\subsubsection{Autocue Operator}
\subsubsection{GFX Operator}
\subsubsection{EVS Operator}
\subsection{Sound Control Room}
\subsubsection{Sound Mixer}
\subsubsection{Gram Operator/Comms Engineer}
This multipurpose position...
\subsection{Vision Control Room}
\section{LTC (Sound Mixing)}
\section{LTC (Colour Grading Suite)}

\end{document}